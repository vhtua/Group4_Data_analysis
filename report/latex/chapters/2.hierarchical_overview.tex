% \section {Overview of Hierarchical clustering methodology}
% Hierarchical clustering is an algorithm that groups similar objects into clusters. The clusters are distinct from each other, and the objects within each cluster are broadly similar to each other.

% Hierarchical clustering is an unsupervised learning technique. This means that a model does not have to be trained, and there is no need for a "target" variable.

% There are two common categories of Hierarchical clustering, agglomerative and diversive. Agglomerative clustering starts in it individual clusters, then merges pair of clusters and continuing until all clusters have been merge in one huge clusters. Diversive clustering is the inverse approach of agglomerative clustering which starts in one cluster then splits into smaller cluster base on their difference. In our project, we only focus on allgomerative clustering  


% The method of hierarchical clustering starts by treating each data point as a separate cluster and then iteratively combines the closest clusters until a stopping criterion is reached. The clusters are visually represented in a hierarchical tree called a dendrogram. 