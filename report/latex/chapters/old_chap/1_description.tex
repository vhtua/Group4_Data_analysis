\section{Project Description}

\begin{itemize}
    \item \textbf{Project Title:} Online Bus Ticket Booking
    
    \item \textbf{Overview:} Online Bus Ticket Booking Web App is a user-friendly web application for VGU students and staff. VGU students as users are provided online QR Bus tickets after booking. The application will also enable VGU staff as administrators to manage and modify all the Information of the bus schedule, and the price of the ticket. Drivers could check the validity of the tickets by scanning QR codes. 

    \begin{enumerate}
        \item As VGU students:
    \begin{enumerate}
        \item[--] They could log in to the system using their student email accounts that are provided by the school organization (VGU).
        \item[--] Students could select the route, and date of the desired bus schedule (Turtle Lake - New Campus route and New Campus - Turtle Lake route).
        \item[--] The registration form is open till 05:30 for Turtle Lake - New Campus Route and 15:30 for New Campus - Turtle Lake Route on that day.
        \item[--] If the students have purchased the ticket for this route, the system notifies "You have already purchased this ticket!" and the students could not purchase the ticket. Otherwise, the system shows the Route Information which consists of the number of available slots and the ticket price. If there are no available slots, the students could not purchase the ticket. 
        \item[--] If there are available slots, students could buy at most 1 ticket/student.
        \item[--] Students could pay for the ticket by using cash or an online payment gateway (Momo).
    \end{enumerate}

        \item As VGU staff (administrators):
    \begin{enumerate}
        \item[--] They could log in to the system using their email accounts and a security code.

        \item[--] Administrators could access the database of the system and then could modify, and update the route, ticket price, and the maximum number of seats per bus.

        \item[--] Administrators could view the information of all student tickets. 
    \end{enumerate}


        \item As bus drivers:
    \begin{enumerate}
        \item[--] They could scan the student QR tickets to manage students to get into the bus. 
    \end{enumerate}
        
    \end{enumerate}
    
      
\end{itemize}