\section{Appendix}
    \subsection{Project Repository on GitHub}
        All source codes, data files, presentations, and survey questions of this project can be found in the project repository on GitHub. Here is the link to this project repository on GitHub: \url{https://github.com/vhtuananh020402/Group4_Data_analysis}


    \subsection{Survey Questions}
        \begin{enumerate}
            \item How old are you? 
            \item What is your gender?
            \item Which area do you work/learn in?
            \item Which movie genre is your favorite? Choose only one.
            \item How often do you watch movies per week?
            \item What is the factor that influences your decision to watch a movie?
            \item Which platform do you use to watch movies?
        \end{enumerate}

    \subsection{R Code}
        \begin{itemize}
            \item dummy\_canberra\_ward.r
        \end{itemize}

        \begin{code}{R}
            # ================== Hierarchical cluster analysis =====================

            # Read the data frame
            df <- read.csv("data/clean_data_v2.csv")
            
            # Omit the NA values of the data frame
            df <- na.omit(df)
            
            # Standardize the Age variable
            df$Age_std <- scale(df$Age)
            
            # Standardize the Frequency variable
            df$Frequency <- factor(
              df$Frequency, 
              order = TRUE,
              levels = c("Less than 2 hours", "2 - 5 hours", "6 - 10 hours", "11 - 15 hours", "16 - 20 hours", "More than 20 hours"))
            df$Frequency_numeric <- as.numeric(factor(df$Frequency))
            df$Frequency_std <- scale(df$Frequency_numeric)
            
            # Turn the nominal variables into dummy variables
            df_dummy <- model.matrix(~ Age_std + Genre + Frequency_std + Field + Factor + Gender + Platform, data = df)
            
            # Calculate the points distance
            point_dist <- dist(df_dummy, method = "canberra")      # Using Canberra distance
            
            # Hierarchical cluster analysis on the data frame
            hc <- hclust(point_dist, method = "ward.D")            # Using Ward's method
            
            # Plot the dendrogram
            dend <- as.dendrogram(hc)
            plot(dend)
            
            # Draw the rectangle around each cluster in k clusters
            k <- 8
            rect.hclust(hc, k, border = 2:8)
            
            
            
            # =================== Gap statistic method to find the optimal number of cluster  ==========
            # Calculate Within-Cluster Dispersion (WCD) for the original data
            wss <- sum(hc$height)
            
            # Generate Random Data for Comparison
            set.seed(123)  # Set seed for reproducibility
            B <- 100  # Number of random datasets
            random_datasets <- lapply(1:B, function(i) matrix(runif(length(df_dummy)), ncol = ncol(df_dummy)))
            
            # Cluster the Random Data
            random_hcs <- lapply(random_datasets, function(random_data) {
              dist_matrix <- dist(random_data, method = dist_method[4])
              hclust(dist_matrix, method = hc_method[1])
            })
            
            
            # Calculate Within-Cluster Dispersion for Random Data
            wss_random <- sapply(random_hcs, function(random_hc) sum(random_hc$height))
            
            
            # Calculate Gap Statistic    
            gap <- (log(wss_random) - log(wss)) + mean(log(wss_random) - log(wss))
            
            # Determine the Optimal Number of Clusters
            num_clusters <- 1 : 10 # Change this range according to the problem domain
            
            gap <- gap[1:length(num_clusters)]
            
            x_labels <- seq(1, length(num_clusters))
            
            plot(num_clusters, gap, xlab = "Number of Clusters", ylab = "Gap Statistic", 
                 main = "Gap Statistic Plot", type = "b")
            
            # Customize x-axis labels
            axis(1, at = x_labels, labels = num_clusters)
            abline(v = 8, col = "red", lty = 2)
            
            
            # ==================== Histogram and observing pattern ====================
            # Add the cluster assignments to the data frame
            df$Cluster <- factor(clusters)
            print(clusters)
            
            # Create a histogram of the Genre distribution in each cluster
            ggplot(df, aes(x = Genre)) +
              geom_histogram(stat = "count", fill = "lightblue", color = "black", linewidth = 0.8) +
              facet_wrap(~ Cluster) +
              theme(axis.text.x = element_text(angle = 45, hjust = 1)) +
              labs(title = "Genre Distribution in Each Cluster", x = "Genre", y = "Count")
            
            # Create a histogram of the Age distribution in each cluster
            ggplot(df, aes(x = Age)) +
              geom_histogram(stat = "count", fill = "orange", color = "black", linewidth = 0.8) +
              facet_wrap(~ Cluster) +
              theme(axis.text.x = element_text(angle = 45, hjust = 1)) +
              labs(title = "Age Distribution in Each Cluster", x = "Age", y = "Count")
            
            # Create a histogram of the Field distribution in each cluster
            ggplot(df, aes(x = Field)) +
              geom_histogram(stat = "count", fill = "red", color = "black", linewidth = 0.8) +
              facet_wrap(~ Cluster) +
              theme(axis.text.x = element_text(angle = 45, hjust = 1)) +
              labs(title = "Field Distribution in Each Cluster", x = "Field", y = "Count")
            
            # Create a histogram of the Frequency distribution in each cluster
            ggplot(df, aes(x = Frequency)) +
              geom_histogram(stat = "count", fill = "darkgrey", color = "black", linewidth = 0.8) +
              facet_wrap(~ Cluster) +
              theme(axis.text.x = element_text(angle = 45, hjust = 1)) +
              labs(title = "Frequency Distribution in Each Cluster", x = "Frequency", y = "Count")
            
            # Create a histogram of the Factor distribution in each cluster
            ggplot(df, aes(x = Factor)) +
              geom_histogram(stat = "count", fill = "lightgreen", color = "black", linewidth = 0.8) +
              facet_wrap(~ Cluster) +
              theme(axis.text.x = element_text(angle = 45, hjust = 1)) +
              labs(title = "Factor Distribution in Each Cluster", x = "Factor", y = "Count")
        \end{code}


         \begin{itemize}
            \item completed\_gower\_ward.r
        \end{itemize}

        \begin{code}{R}
            library(cluster)      # contain function 'daisy'
            library(factoextra)   # clustering visualization
            library(ggplot2)      # draw distribution graph
            
                      # --- Data Preparation --- #
            df <- read.csv("data/clean_data_v2.csv")
            df <- na.omit(df)     # remove missing values in dataset
            
            # standardize age, then add into numerical value
            df$Age_std <- scale(df$Age)
            num_attr <- c("Age")
            
            # add into categorical value
            cat_attr <- c("Field", "Genre", "Factor")
            df[cat_attr] <- lapply(df[cat_attr], as.factor)
            
            # add into ordinal value
            ord_attr <- c("Frequency")
            df$Frequency <- factor(df$Frequency, 
                                   order = TRUE, 
                                   level = c("Less than 2 hours", 
                                             "2 - 5 hours", 
                                             "6 - 10 hours", 
                                             "11 - 15 hours", 
                                             "16 - 20 hours", 
                                             "More than 20 hours"))
            
            # put everything into a complete dataset
            process_dataset <- df %>% select(num_attr, ord_attr, cat_attr)
            
            head(process_dataset)
            
                      # --- Calculation --- #
            # calculate Gower's distance
            gower_dist <- daisy(process_dataset, metric="gower")   
            
            # hierarchical clustering, using ward.D method 
            gower_hcl <- hclust(gower_dist, method = "ward.D")  
            
                      # --- DENDROGRAM ---- #
            # plot dendrogram
            plot(gower_hcl, cex = 0.6)
            
            # draw borders for the individual clusters
            rect.hclust(gower_hcl, k = 7, border = 2:7)
            
                      # --- HISTOGRAM --- #
            # cut into k clusters
            k <- 7
            clusters <- cutree(gower_hcl, k)
            
            # add the cluster assignments to the data frame
            df$Cluster <- factor(clusters)
            
            # histogram of Genre distribution in each cluster
            ggplot(df, aes(x = Genre)) +
              geom_histogram(stat = "count", fill = "lightblue", color = "black", linewidth = 0.8) +
              facet_wrap(~ Cluster) +
              theme(axis.text.x = element_text(angle = 45, hjust = 1)) +
              labs(title = "Genre Distribution in Each Cluster", x = "Genre", y = "Count")
            
            # histogram of Field distribution in each cluster
            ggplot(df, aes(x = Field)) +
              geom_histogram(stat = "count", fill = "red", color = "black", linewidth = 0.8) +
              facet_wrap(~ Cluster) +
              theme(axis.text.x = element_text(angle = 45, hjust = 1)) +
              labs(title = "Field Distribution in Each Cluster", x = "Field", y = "Count")
            
            # histogram of Factor distribution in each cluster
            ggplot(df, aes(x = Factor)) +
              geom_histogram(stat = "count", fill = "lightgreen", color = "black", linewidth = 0.8) +
              facet_wrap(~ Cluster) +
              theme(axis.text.x = element_text(angle = 45, hjust = 1)) +
              labs(title = "Factor Distribution in Each Cluster", x = "Factor", y = "Count")
            
            # create a histogram of Age distribution in each cluster
            ggplot(df, aes(x = Age)) +
              geom_histogram(stat = "count", fill = "orange", color = "black", linewidth = 0.8) +
              facet_wrap(~ Cluster) +
              theme(axis.text.x = element_text(angle = 45, hjust = 1)) +
              labs(title = "Age Distribution in Each Cluster", x = "Age", y = "Count")
            
            # create a histogram of the Genre distribution in each cluster
            ggplot(df, aes(x = Frequency)) +
              geom_histogram(stat = "count", fill = "darkgrey", color = "black", linewidth = 0.8) +
              facet_wrap(~ Cluster) +
              theme(axis.text.x = element_text(angle = 45, hjust = 1)) +
              labs(title = "Frequency Distribution in Each Cluster", x = "Frequency", y = "Count")
            
        \end{code}